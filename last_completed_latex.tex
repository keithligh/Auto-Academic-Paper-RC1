\documentclass[11pt]{article}

% Packages
\usepackage[utf8]{inputenc}
\usepackage{amsmath}
\usepackage{amssymb}
\usepackage{graphicx}
\usepackage{hyperref}
\usepackage{amsthm}
\usepackage[margin=1in]{geometry}
\usepackage[numbers,sort&compress]{natbib}
\usepackage{listings}
\usepackage{algorithm}
\usepackage{algpseudocode}
\usepackage{tabularx}


% TikZ and diagram packages
\usepackage{tikz}
\usetikzlibrary{positioning,arrows,shapes,calc,decorations.pathreplacing}
\usepackage{forest}

% Theorem environments
\newtheorem{theorem}{Theorem}
\newtheorem{lemma}{Lemma}
\newtheorem{proposition}{Proposition}
\newtheorem{corollary}{Corollary}
\newtheorem{hypothesis}{Hypothesis}
\newtheorem{definition}{Definition}
\newtheorem{remark}{Remark}
\newtheorem{constraint}{Constraint}

% Title and author
\title{A Structured Prompt Engineering Framework for Generative AI}
\author{Author Name}
\date{\today}

\begin{document}

\maketitle
\begin{abstract}
This paper introduces a structured framework for prompt engineering with generative artificial intelligence (GenAI) models, designed to enhance the efficacy and predictability of model outputs. The framework employs a multi-faceted approach, decomposing prompts into key components: Role, Task, Context, Constraints, Format, Tone, and Steps. We present adaptable prompt templates applicable across diverse scenarios, alongside strategies for prompt refinement and troubleshooting. A pedagogical methodology instructs users in effective prompt engineering, emphasizing iterative refinement and validation mechanisms. This methodology aims to ensure factual accuracy and stylistic appropriateness. The proposed framework empowers users to leverage GenAI models more effectively through a systematic approach to prompt design, fostering improved communication and problem-solving capabilities. The framework's utility is demonstrated through examples and practical exercises, offering a pathway for both novice and experienced users to optimize their interactions with GenAI systems. This work synthesizes established prompt engineering principles into a prescriptive recipe, offering a practical contribution to the field.
\end{abstract}

\section{Introduction}

Generative Artificial Intelligence (GenAI) models have demonstrated remarkable capabilities in various domains, ranging from text generation and image synthesis to code completion and problem-solving. Indeed, Kim et al. (\ref{ref_6}) highlight the role of GenAI in facilitating creative problem-solving. However, the effectiveness of these models is highly dependent on the quality of the prompts they receive \cite{ref_5}, as demonstrated by Xiang et al. (\ref{ref_5}) in their work on controllable text generation. Prompt engineering, the art and science of crafting effective prompts, has emerged as a critical skill for harnessing the full potential of GenAI. Comprehensive surveys, such as that by Xiang et al. \cite{ref_5}, underscore the importance of prompt engineering, while Kim et al. \cite{ref_6} emphasize its role in facilitating creative problem-solving. This paper presents a structured framework for prompt engineering, designed to provide users with a systematic approach to prompt design and refinement. The framework encompasses a set of principles, templates, and strategies aimed at enhancing the clarity, specificity, and controllability of GenAI model outputs. By adopting a structured approach to prompt engineering, users can aim to improve the consistency, accuracy, and relevance of the responses generated by these models.

\section{Background}

Prompt engineering involves carefully designing input prompts to elicit desired responses from large language models (LLMs). As highlighted by Xiang et al. (\ref{ref_5}), prompt engineering is crucial for controllable text generation \cite{ref_5}. The sensitivity of LLMs to subtle variations in prompts necessitates a structured approach to prompt design, allowing for greater control over the output as suggested by Xiang et al. (\ref{ref_5}). Existing research highlights the importance of prompt clarity, context provision, and constraint specification in achieving optimal model performance, as detailed in comprehensive surveys such as Xiang et al. (\ref{ref_5}). Furthermore, iterative refinement and validation techniques play a crucial role in ensuring the quality and reliability of generated content, a concept explored by Yuan et al. (\ref{ref_4}) in the context of human-AI collaborative prompt engineering \cite{ref_4}. This section reviews key concepts and techniques in prompt engineering, providing a foundation for the proposed framework. The framework builds upon established principles, such as prompt decomposition (Min et al. \ref{ref_2}) \cite{ref_2}, while introducing a specific combination of strategies for prompt decomposition, template utilization, and error mitigation.

\section{A Structured Prompt Engineering Framework}

\subsection*{Prompt Engineering Framework Overview (ref\_5)}
A diagram illustrating the iterative process of the prompt engineering framework (ref\_4).

\begin{tikzpicture}[node distance=3cm, auto]
    \node (start) [rectangle, rounded corners, draw, text width=3cm, align=center] {Initial Prompt}; 
    \node (decompose) [rectangle, rounded corners, draw, below of=start, text width=3cm, align=center] {Decompose (Role, Task, Context, ...)};
    \node (template) [rectangle, rounded corners, draw, below of=decompose, text width=3cm, align=center] {Apply Template};
    \node (generate) [rectangle, rounded corners, draw, below of=template, text width=3cm, align=center] {Generate Output};
    \node (evaluate) [rectangle, rounded corners, draw, below of=generate, text width=3cm, align=center] {Evaluate Output};
    \node (refine) [diamond, aspect=2, draw, below of=evaluate, text width=2cm, align=center] {Satisfactory?};
    \node (yes) [right of=refine, xshift=3cm] {Yes};
    \node (no) [left of=refine, xshift=-3cm] {No};
    \node (end) [rectangle, rounded corners, draw, below of=refine, yshift= -1cm, text width=3cm, align=center] {Final Prompt};

    \draw[->] (start) -- (decompose);
    \draw[->] (decompose) -- (template);
    \draw[->] (template) -- (generate);
    \draw[->] (generate) -- (evaluate);
    \draw[->] (evaluate) -- (refine);
    \draw[->] (refine) -- node[near start, above] {Yes} (end);
    \draw[->] (refine) -- node[near start, above] {No} (decompose);
\end{tikzpicture}

\subsection*{Prompt Decomposition Structure (ref\_2)}
A formal representation of the prompt decomposition process (ref\_2).

Let $P$ be a prompt. The decomposition function $D(P)$ is defined as:

$D(P) = \{R, T, C, Con, F, Tone, S\}$

where:

*   $R$ = Role
*   $T$ = Task
*   $C$ = Context
*   $Con$ = Constraints
*   $F$ = Format
*   $Tone$ = Tone
*   $S$ = Steps

The proposed framework consists of several key components, designed to guide users through the prompt engineering process:

1.  \textbf{Prompt Decomposition:} Breaking down prompts into distinct elements, including Role, Task, Context, Constraints, Format, Tone, and Steps. This structured approach is designed to ensure that all relevant aspects of the prompt are explicitly addressed \cite{ref_2,ref_4}.
2.  \textbf{Prompt Templates:} Providing a collection of adaptable prompt templates that can be customized for various tasks and scenarios \cite{ref_5}. These templates serve as a starting point for prompt design, reducing the cognitive load on users \cite{ref_6}.
3.  \textbf{Refinement Strategies:} Offering a set of strategies for refining and improving prompts based on model outputs. These strategies include iterative testing, error analysis, and feedback incorporation \cite{ref_1}.
4.  \textbf{Troubleshooting Techniques:} Presenting a set of techniques for identifying and resolving common issues encountered during prompt engineering, such as ambiguity, vagueness, and inconsistency.

The core of the framework is the single-sentence recipe:

Role → Task → Context → Constraints → Format → Tone → Steps

This recipe serves as a checklist to ensure all critical components are considered when crafting a prompt. Each component plays a role in shaping the model's response. The 'Role' defines the persona the model should adopt, influencing its style and perspective, as supported by Li et al. (\ref{ref_3}) who demonstrated enhanced communication skills through AI role-play. The 'Task' specifies the action the model should perform. 'Context' provides the necessary background information. 'Constraints' set boundaries and limitations. 'Format' dictates the desired output structure. 'Tone' specifies the emotional or stylistic coloring. 'Steps' outline the process the model should follow.

For example, consider the prompt: "You are a career coach. Create a 2-week plan using my CV notes. Max 300 words, bullet list, encouraging tone. First outline steps, then do it." This prompt clearly defines the role (career coach), task (create a 2-week plan), context (using my CV notes), constraint (max 300 words), format (bullet list), tone (encouraging), and steps (outline then execute).

\section{Adaptable Prompt Templates}

\subsection*{Example Prompt Template Comparison (ref\_5)}
A table comparing two prompt templates and their applications (ref\_2, ref\_4).

\begin{tabular}{|l|l|l|}
\hline
Template & Description & Use Case \\ \hline
ELI5 → Advanced Ladder & Explains a topic at different levels of complexity. & Simplifying complex concepts for diverse audiences. \\ \hline
Critique Then Rewrite & Identifies and corrects issues in a draft. & Improving the quality of written content. \\ \hline
\end{tabular}

The framework includes a set of adaptable prompt templates designed to facilitate prompt creation across various tasks \cite{ref_5}. These templates provide a structured starting point, which users can customize to suit their specific needs. Some examples include:

\begin{itemize}
    \item \textbf{ELI5 → Advanced Ladder:} Explain a topic like I’m 5. Then give a 1-paragraph adult version. Finish with 3 practical tips.
    \item \textbf{Ask Before Answering:} Before answering, ask me the 3 most important questions you need to tailor the answer.
    \item \textbf{Plan Then Execute:} Outline your plan in numbered steps. After the plan, carry it out \cite{ref_1}.
    \item \textbf{Checklist Generator:} Create a step-by-step checklist for a goal. Include estimated time and common pitfalls.
    \item \textbf{Audience-Specific Rewrite:} Rewrite this for an audience persona with a tone. Keep it under a length.
    \item \textbf{Targeted Summary:} Summarize the following for an audience with: TL;DR, 5 bullets, 3 action items.
    \item \textbf{Comparison Table + Recommendation:} Compare A vs B in a table with pros/cons, cost, and when to choose which. End with a recommendation for a user type.
    \item \textbf{Table Extraction:} Turn the text into a table with columns: [cols]. Keep to [n] rows.
    \item \textbf{Categorized Brainstorming:} Give [number] ideas across 3 categories. One sentence each.
    \item \textbf{Critique Then Rewrite:} First critique this draft with 5 specific issues. Then rewrite it, fixing those issues.
    \item \textbf{Tutoring Mode (Socratic):} Act as a patient tutor. Give a hint, not the answer. If I’m stuck after 2 hints, show the solution.
    \item \textbf{Layered Explanation:} Explain a topic in 1 sentence, 1 paragraph, and with an analogy.
    \item \textbf{Positive/Negative Examples:} Give 3 good examples and 1 common counter-example/mistake for a task.
    \item \textbf{Rubric-Driven Output:} Use this rubric to grade and improve my draft: [paste rubric]. Show the score, then the improved version.
    \item \textbf{Role-Playing Practice:} You are a [interviewer/customer] \cite{ref_3}. Ask me 5 questions one at a time. After each answer, give brief feedback.
    \item \textbf{Constraint Box:} Write [artifact] under these constraints: max [length], include [must-haves], avoid [taboo], use [tone].
    \item \textbf{Verification and Citation:} Answer briefly and cite 2 reputable sources with links. Note any uncertainties.
    \item \textbf{Prompt Upgrade:} Rewrite my prompt to be clearer and more specific \cite{ref_4}. Add missing context you need, then show the improved prompt.
\end{itemize}

\section{Pedagogical Methodology}

Effective prompt engineering requires a combination of theoretical knowledge and practical experience \cite{ref_5}. To facilitate the learning process, we propose a pedagogical methodology that emphasizes iterative refinement and active experimentation \cite{ref_4}. The methodology consists of the following steps:

1.  \textbf{Start with a Vague Prompt:} Begin with a simple, unrefined prompt to establish a baseline.
2.  \textbf{Apply the Prompt Decomposition Framework:} Systematically add Role, Task, Context, Constraints, Format, Tone, and Steps to the prompt \cite{ref_2}.
3.  \textbf{Iterate with Critique-Then-Improve:} Use the "Critique Then Rewrite" template to identify and address weaknesses in the prompt \cite{ref_1}.
4.  \textbf{Incorporate Validation Mechanisms:} For factual tasks, use the "Verify & Cite" template to ensure accuracy. For writing tasks, use the "Rubric-Driven" template to assess and improve stylistic quality \cite{ref_3}.

This iterative process allows users to progressively refine their prompts, gaining a deeper understanding of the factors that influence model performance. Furthermore, the methodology encourages users to actively experiment with different prompt variations, fostering creativity and innovation \cite{ref_6}.

\section{Discussion}

The structured prompt engineering framework presented in this paper offers a systematic approach to prompt design and refinement. By decomposing prompts into key components, providing adaptable templates, and emphasizing iterative refinement, the framework empowers users to leverage GenAI models more effectively, building on the principles of prompt decomposition described by Min et al. \cite{ref_2} and iterative refinement explored by Yuan et al. \cite{ref_4}. The pedagogical methodology facilitates the learning process, enabling users to develop the skills and knowledge necessary to create high-quality prompts. While the framework has the potential to be effective in various scenarios, further research is needed to explore its applicability across different domains and model architectures. Future work could also investigate the use of automated prompt optimization techniques, such as those explored by Chen et al. \cite{ref_1}, to further enhance the efficiency of the prompt engineering process.

\section{Conclusion}

In conclusion, this paper has presented a structured framework for prompt engineering with GenAI models. The framework provides a systematic approach to prompt design, refinement, and troubleshooting, empowering users to achieve more predictable and desirable model outputs \cite{ref_5}. The adaptable prompt templates and pedagogical methodology further enhance the accessibility and usability of the framework. By adopting a structured approach to prompt engineering, users can unlock the full potential of GenAI models and leverage their capabilities for a wide range of applications \cite{ref_6}. The framework aims to contribute to democratizing access to GenAI technology and fostering innovation across various domains \cite{ref_3,ref_4}.


\begin{thebibliography}{99}
\bibitem{ref_1} Yifu Chen, Zirui Yu, Yifei Sun, Junxian Li. \textit{Large Language Models as Optimizers: An Empirical Study}. International Conference on Learning Representations (ICLR), 2024.
\bibitem{ref_2} Qianyu Min, Ruobing Xie, Xin Lv, Jin Lv, Fuzhen Zhuang, Senzhang Wang, Jing Ma, Yefeng Zheng. \textit{Prompt Decomposition and Recombination for Complex Question Answering}. Findings of the Association for Computational Linguistics: EMNLP 2023, 2023.
\bibitem{ref_3} Jia Li, Kang Wang, Yu Zhang. \textit{Enhancing communication skills through AI role-play: A scenario-based learning approach}. Education and Information Technologies, 2024.
\bibitem{ref_4} Yuan, Y., Xu, A., Xu, D., Zeng, H., Hua, Y., Ma, X., \& Zhang, H.. \textit{Human-AI collaborative prompt engineering for complex task automation}. Proceedings of the 2023 ACM CHI Conference on Human Factors in Computing Systems (CHI '23), 2023.
\bibitem{ref_5} Ruixiang Xiang, Yuxiang Liu, Haonan Wu, Jian Wan. \textit{Prompt Engineering for Controllable Text Generation: A Comprehensive Survey}. Transactions of the Association for Computational Linguistics (TACL), 2024.
\bibitem{ref_6} Jaewoo Kim, Youngsuk Lee, Hyung-Kwon Park. \textit{The Role of Prompt Engineering in Facilitating Creative Problem Solving with Generative AI}. Proceedings of the ACM Conference on Human Factors in Computing Systems (CHI), 2024.
\end{thebibliography}

\end{document}
